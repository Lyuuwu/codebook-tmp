% DP 優化技巧筆記(何時可用、條件、複雜度)

\noindent
\textbf{Monotonic Queue(單調隊列)}
\begin{itemize}
  \item 適用情境:轉移為固定寬度窗口的 min/max,例如 $\mathrm{dp}[i]=\mathrm{cost}[i]+\min\limits_{i-W\le j<i} \mathrm{dp}[j]$。
  \item 條件:窗口僅往右滑動;可用雙端佇列維護候選(按值單調)。
  \item 複雜度:$O(n)$。
  \item 範例程式:`Contents/DP/monoqueue_optimization.cpp`。
\end{itemize}

\noindent
\textbf{Convex Hull Trick(CHT)/ Li Chao Tree}
\begin{itemize}
  \item 目標型態:$\mathrm{dp}[i]=\min\limits_j (m_j\,x_i+b_j)$ 或取最大值的線性包絡。
  \item Li Chao Tree:對 $x$ 有界時通用,無需斜率與查詢單調;單次操作 $O(\log \text{range})$。
  \item 單調 CHT:若斜率插入單調且查詢 $x$ 單調,可用雙端隊列 $O(1)$ 攤還。
  \item 範例程式:`Contents/DP/li_chao_tree.cpp`(Min 版本)。
\end{itemize}

\noindent
\textbf{Divide and Conquer DP 優化}
\begin{itemize}
  \item 型態:$\mathrm{dp}[k][i]=\min\limits_{j<i}\{\mathrm{dp}[k-1][j]+C(j,i)\}$。
  \item 條件:存在最優決策單調性 $\operatorname{opt}[k][i]\le \operatorname{opt}[k][i+1]$。
  \item 複雜度:$O(KN\log N)$ 或 $O(KN)$ 視成本計算而定。
  \item 範例骨架:`Contents/DP/dnc_dp_optimization.cpp`。
\end{itemize}

\noindent
\textbf{Knuth 優化(區間 DP)}
\begin{itemize}
  \item 型態:$\mathrm{dp}[l][r]=\min\limits_{l\le k<r}\{\mathrm{dp}[l][k]+\mathrm{dp}[k+1][r]\}+w(l,r)$。
  \item 條件:$w$ 滿足四邊形不等式,且 $\operatorname{opt}[l][r-1]\le \operatorname{opt}[l][r]\le \operatorname{opt}[l+1][r]$。
  \item 常見:合併石子、最優二叉搜尋樹、矩陣鏈變形。
  \item 複雜度:$O(n^2)$。
  \item 範例骨架:`Contents/DP/knuth_optimization.cpp`。
\end{itemize}

\noindent
\textbf{Bitset / SOS / 參數二分(Alien Trick)}
\begin{itemize}
  \item Bitset:子集和/背包可用位運算加速,常達 $O(\frac{n W}{w})$,$w$ 為機器字大小。
  \item SOS DP:對子集函數作快速轉換(FWT/子集捲積),用於子集枚舉與捲積類 DP。
  \item Alien Trick:在答案上二分,內層用 DP 檢驗可行性,配合 Lagrange 乘子式轉換權重。
\end{itemize}

\noindent
\textbf{何時選用哪一種?}
\begin{itemize}
  \item 固定窗口取極值:單調隊列。
  \item 線性包絡:斜率截距線性轉移 → CHT/Li Chao。
  \item 多段/分組轉移且 argmin 單調:D\&C 優化。
  \item 區間型兩段分裂 + $w(l,r)$ 滿足四邊形不等式:Knuth 優化。
  \item 子集/位元 DP:Bitset 或 SOS。
\end{itemize}
