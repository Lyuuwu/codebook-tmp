% 指數與取模(mod)之間的關係與常用公式

\noindent
\textbf{Euler 定理與指數化簡}
\[
\gcd(a,m)=1 \;\Rightarrow\; a^{\varphi(m)} \equiv 1 \pmod m.
\]
設 $e \equiv b \pmod{\varphi(m)}$。若 $\gcd(a,m)=1$ 且 $b$ 很大,常用以下化簡:
\[
a^b \equiv a^{e} \pmod m\quad(\text{若 } e=0\text{,可取 } e:=\varphi(m)).
\]

\noindent
\textbf{Carmichael 函數與更強化簡}
\[
\text{對所有互質 } a,m:\; a^{\lambda(m)} \equiv 1 \pmod m,\quad x\equiv y\ (\bmod\ \lambda(m))\Rightarrow a^x\equiv a^y\ (\bmod\ m).
\]
分解 $m=\prod p_i^{k_i}$ 後
\[
\lambda(m)=\operatorname{lcm}\bigl(\lambda(p_1^{k_1}),\dots,\lambda(p_r^{k_r})\bigr),\quad
\lambda(p^k)=\begin{cases}
2^{k-2}, & p=2,\ k\ge 3,\\
\varphi(p^k), & \text{其餘情形}.
\end{cases}
\]
實務上,若 $\gcd(a,m)=1$,可把指數先對 $\lambda(m)$(或 $\varphi(m)$)取模,再用上式處理 $e=0$ 的情形。

\noindent
\textbf{乘法階(Multiplicative Order)}
\[
\operatorname{ord}_m(a)=\min\{t>0: a^t\equiv 1\ (\bmod\ m)\},\quad \operatorname{ord}_m(a)\mid \lambda(m)\mid \varphi(m).
\]
常用判別:$a^x\equiv a^y\ (\bmod\ m)$ 當且僅當 $x\equiv y\ (\bmod\ \operatorname{ord}_m(a))$(假設 $\gcd(a,m)=1$)。

\noindent
\textbf{模指數運算快速規則}
\begin{align*}
(a^x\bmod m)(a^y\bmod m) &\equiv a^{x+y}\pmod m,\\
(a^x)^y &\equiv a^{xy}\pmod m,\\
\gcd(a,m)=1,\; x\equiv y\ (\bmod\ \varphi(m)) &\Rightarrow a^x\equiv a^y\pmod m,\\
\text{$p$ 為質數且 } \gcd(a,p)=1:&\; a^{k(p-1)+r}\equiv a^r\pmod p.
\end{align*}
若 $m=\prod p_i^{k_i}$,計算 $a^b\bmod m$ 可先各自求 $a^b\bmod p_i^{k_i}$(遇到不互質時可配合質因數分解與指數提取),再用中國剩餘定理合併。

