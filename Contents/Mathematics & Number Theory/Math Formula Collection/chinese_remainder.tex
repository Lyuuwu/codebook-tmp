\subsubsection*{Chinese Remainder Theorem}
設同餘系統
\[
x \equiv a_i \pmod{m_i} \quad (i = 1,\dots,k),
\]
其中 $m_i$ 兩兩互質,令 $M = \prod_{i=1}^k m_i$,$M_i = M/m_i$,再取 $t_i \equiv M_i^{-1} \pmod{m_i}$。則唯一解(模 $M$)為
\[
x \equiv \sum_{i=1}^k a_i M_i t_i \pmod{M}.
\]

\paragraph{兩式合併(允許非互質)}
\begin{lstlisting}[language=C++,numbers=left]
// solve x ≡ a1 (mod m1), x ≡ a2 (mod m2)
// return {x0, lcm}; if no solution, lcm = -1
pair<ll, ll> crt(ll a1, ll m1,
    ll a2, ll m2) {
    ll g = std::gcd(m1, m2);
    if ((a2 - a1) % g != 0) return {0, -1}; // no solution

    ll lcm = m1 / g * m2;
    ll m1_reduced = m1 / g;
    ll m2_reduced = m2 / g;

    ll diff = (a2 - a1) / g % m2_reduced;
    if (diff < 0) diff += m2_reduced;

    ll inv = mod_pow(m1_reduced, m2_reduced - 1, m2_reduced);
    ll step = diff * inv % m2_reduced;
    ll x0 = (a1 + step * m1) % lcm;
    if (x0 < 0) x0 += lcm;
    return {x0, lcm};
}
\end{lstlisting}
遞增地將每個同餘式與當前解做合併即可取得最終答案,也能偵測無解情況。
